
\begin{frame}
{\huge subsidiary}
\begin{center}
\begin{enumerate}\Large
  \item \textbf{次要的}
\end{enumerate}
\end{center}
\end{frame}
