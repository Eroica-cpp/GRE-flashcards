
\begin{frame}
\begin{center}
{\fontsize{2.5cm}{1em}\selectfont ragged}
\end{center}
\end{frame}
